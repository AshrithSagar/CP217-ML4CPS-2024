\section*{Problem 3}

\textbf{(Gradient descent, L2 regularization, and variational calculus)}
\begin{enumerate}[label= (\alph*), noitemsep, topsep=0pt]
      \item Show that a first-order gradient descent always leads to a decrease in the function value for smooth functions.

      \item Show that L2 regularization leads to a ``decay'' term in the gradient descent.
            What is the term?

      \item (Bishop, Exercise problem 1.25) Taking motivation for the result 1.89 in Bishop, prove the same result for multidimensional case, as given in exercise problem 1.25.\\
            P.S If interested, you can look up exercise problem 1.27 of Bishop.
\end{enumerate}

\subsection*{Solution}

\subsubsection*{(a) First-order gradient descent for smooth functions}

The gradient descent update rule is given by
\begin{equation*}
      \theta_{t+1} = \theta_t - \eta \nabla f(\theta_t)
\end{equation*}
where \(\theta_t\) is the parameter at time \(t\), \( \eta > 0 \) is the learning rate, and \(f(\theta)\) is the function to be minimized.
For smooth functions, the gradient of the function at a point gives the direction of steepest ascent, thereby the negative of the gradient gives the direction of steepest descent.
The smoothness property of the function ensures that the gradient is continuous and differentiable, ensuring that the function value decreases at each iteration, i.e., \(f(\theta_{t+1}) < f(\theta_t)\).

\subsubsection*{(b) L2 regularization and the ``decay'' term}

The L2 regularisation term is given by
\begin{equation*}
      \mathcal{L}(\theta) = f(\theta) + \frac{\lambda}{2} \| \theta \|_2^2
\end{equation*}

The gradient descent update rule with L2 regularization is given by
\begin{equation*}
      \theta_{t+1} = \theta_t - \eta \left( \nabla f(\theta_t) + \lambda \theta_t \right)
\end{equation*}
This is the term that occurs in the L2 patametrised gradient descent.

\subsubsection*{(c) Multidimensional case}
