\section*{Problem 4}

\textbf{Spectral Clustering}\\
In this question, we will explore the properties of the Graph Laplacian, \( \mathcal{L} \) used in spectral clustering.

\begin{enumerate}[label= (\alph*), noitemsep, topsep=0pt]
    \item With the definition of the Graph Laplacian, \( \mathcal{L} \) discussed in class, show that for any vector \( f \in \mathbb{R}^{n} \) we have
          \[
              f^{T} L f=\frac{1}{2} \sum_{i, j=1}^{n} w_{i j}\left(f_{i}-f_{j}\right)^{2}
          \]
          where \( w_{i j} \) are elements of the weight matrix \( \mathcal{W} \)

    \item Prove that \( \mathcal{L} \) is positive semi-definite.

    \item Prove that the smallest eigenvalue of \( \mathcal{L} \) is 0 and find the corresponding eigenvector.

    \item Show that \( \mathcal{L} \) has \( n \) non-negative, real-valued eigenvalues, \( 0 \leq \lambda_{1} \leq \lambda_{2} \ldots \leq \lambda_{n} \).

    \item Show that the number of connected components in a graph G is the multiplicity k of the eigenvalue 0 for the Graph Laplacian, \( \mathcal{L} \).\\
          P.S --- Refer to the pdf on spectral clustering by Chunpai Wang, uploaded in the files section of Week-12 channel for any doubts.
\end{enumerate}

\subsection*{Solution}

\subsubsection*{(a) Graph Laplacian \( \mathcal{L} \)}

The graph Laplacian matrix \( \mathcal{L} \) is defined as \( \mathcal{L} = \mathcal{D} - \mathcal{W} \), where \( \mathcal{D} \) is the degree matrix and \( \mathcal{W} \) is the weight matrix.
Given a vector \( f \in \mathbb{R}^{n} \), we have
\begin{align*}
    f^{T} L f
     & =
    f^{T} (\mathcal{D} - \mathcal{W}) f
    = f^{T} \mathcal{D} f - f^{T} \mathcal{W} f
    =
    \sum_{i=1}^{n} d_{i} f_{i}^{2} - \sum_{i, j=1}^{n} w_{i j} f_{i} f_{j}
    \\ & =
    \frac{1}{2} \left( \sum_{i=1}^{n} d_{i} f_{i}^{2} - 2 \sum_{i, j=1}^{n} w_{i j} f_{i} f_{j} + \sum_{j=1}^{n} d_{j} f_{j}^{2} \right)
    =
    \frac{1}{2} \sum_{i, j=1}^{n} w_{i j} (f_{i} - f_{j})^{2}
\end{align*}

Hence, we have the result.

\subsubsection*{(b) Positive Semi-definiteness of \( \mathcal{L} \)}

The elements of the weight matrix \( \mathcal{W} \) are non-negative, and the degree matrix \( \mathcal{D} \) is a diagonal matrix with non-negative elements.
Hence, we have that \( \mathcal{L} \) is positive semi-definite.

\subsubsection*{(c) Smallest Eigenvalue and Eigenvector of \( \mathcal{L} \)}

The smallest eigenvalue of \( \mathcal{L} \) is 0 because the diagonal elements of \( \mathcal{D} \) are all 0.
The corresponding eigenvector can be found out to be the all-ones vector \( \mathbf{1} \).

\subsubsection*{(d) Non-negative, Real-valued Eigenvalues of \( \mathcal{L} \)}

Since \( \mathcal{L} \) is positive semi-definite, all its eigenvalues are non-negative.
Further, the eigenvalues of \( \mathcal{L} \) are real-valued because \( \mathcal{L} \) is symmetric.
Hence, we have \( 0 \leq \lambda_{1} \leq \lambda_{2} \ldots \leq \lambda_{n} \).
