\section*{Problem 4}

\textbf{Spectral Clustering}\\
In this question, we will explore the properties of the Graph Laplacian, \( \mathcal{L} \) used in spectral clustering.

\begin{enumerate}[label= (\alph*), noitemsep, topsep=0pt]
    \item With the definition of the Graph Laplacian, \( \mathcal{L} \) discussed in class, show that for any vector \( f \in \mathbb{R}^{n} \) we have
          \[
              f^{T} L f=\frac{1}{2} \sum_{i, j=1}^{n} w_{i j}\left(f_{i}-f_{j}\right)^{2}
          \]
          where \( w_{i j} \) are elements of the weight matrix \( \mathcal{W} \)

    \item Prove that \( \mathcal{L} \) is positive semi-definite.

    \item Prove that the smallest eigenvalue of \( \mathcal{L} \) is 0 and find the corresponding eigenvector.

    \item Show that \( \mathcal{L} \) has \( n \) non-negative, real-valued eigenvalues, \( 0 \leq \lambda_{1} \leq \lambda_{2} \ldots \leq \lambda_{n} \).

    \item Show that the number of connected components in a graph G is the multiplicity k of the eigenvalue 0 for the Graph Laplacian, \( \mathcal{L} \).\\
          P.S --- Refer to the pdf on spectral clustering by Chunpai Wang, uploaded in the files section of Week-12 channel for any doubts.
\end{enumerate}

\subsection*{Solution}
